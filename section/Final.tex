\section*{Statutory Interpretation}
\begin{itemize}
    \item Parties
    \item Capacity/Authority
        \begin{description}
            \item[Authority:] Do they have the authority to do this? Court hierarchy. 
            \item[Capacity:] age, mental issues.  
        \end{description}
    \item Jurisdiction
        \begin{description}
            \item[legislation:] state or cth
            \item[location:] Where does the aggression happen? 
        \end{description}
    \item Time/Commencement
        \begin{description}
            \item[Time:]Royal Assent (Cth 28 days after)
        \end{description}
    \item Identify the legal issues
        \begin{itemize}
            \item Party One 
                \begin{description}
                    \item[Sole Party:] list all different offenses.  
                \end{description}
            \item Party Two 
            \item Formation of issues\
                \begin{description}
                    \item[Forming:] As a question + id s legs.  
                \end{description}
        \end{itemize}
    \item SI principles. 
    \begin{itemize}
            \item Principle: AIA(Cth)s 15AA; AIA(Qld) s14A(1)
                \begin{description}
                    \item[Propulsive:] CIC v Bankstown; Bluesky 
                \end{description}
    \item Identify the ambiguities
        \begin{itemize}
            \item Text \& Context
                \begin{description}
                    \item[Context:] Purpose of the act.
                    \item[E.g.] Did xxxxxxxxx 
                \end{description}
        \end{itemize}
    \item Apply the Law to the Facts
    \item Conclusion
\end{itemize}

\section*{Jurisprudence}
Jurisprudence has two components: judgecraft and philosophy

\subsection*{Answering Structure}
\begin{itemize}
    \item Introduction
        \begin{itemize}
            \item concepts in brief
            \item introduces the differences
        \end{itemize}
\end{itemize}

\section*{First Nation Question}

