\section*{Workshop One}


\section*{Workshop Two \\ Making and Finding Law -- Legislation}
\date{March. 12th, 2024}

\subsection*{Useful Information}
\begin{itemize}
    \item This website will have some legal cases available\\ \href{https://www.ruleoflaw.org.au/}{Rule of Law Education Center}.
    \item Nowadays, consider the pronouns. In QCAT, we use commissioner/member for some positions. 
    \begin{description}
     			\item[Note:] a
         	\end{description}

\end{itemize}

\subsection*{Term}
\begin{itemize}
    \item language is important to reflect the mind. 
    \item Provenance: 
        \begin{itemize}
            \item authoritative materials
            \item up-to-date
            \item certainty: authoritative
            \item free
        \end{itemize}
    \item Explanatory Notes/memoranda
        \begin{description}
     		\item[Note:] Explanatory Notes for QLD; Explanatory Memoranda for Common Wealth. 
                \item [Exam Item:] Need to learn to locate this information
         \end{description}
    \item Parliamentary debates
        \begin{itemize}
            \item What was said/argued etc in Parliament
                \begin{description}
                    \item[Note:] This explains why we have the legislation and purpose. Second read or Explanatory Notes.
                \end{description}
            \item Records of the debates are referred to as 'Hansard.'
        \end{itemize}
    \item Abbreviations in law
        \begin{itemize}
            \item jurisdiction: (QLD), (Cth)
            \item K - contract
            \item cf - comparison 
            \item V, eg. Can V Smith, 
                \begin{description}
                    \item[Criminal:]
                    \item[Civil:]
                \end{description}
            \item Cth -- commonwealth (Australian federal)
        \end{itemize}
          
\end{itemize}

\subsection*{Common Mistakes}
\begin{itemize}
    \item 
\end{itemize}

\subsection*{Legal Research Skills}




\subsection*{Legal Process}

\subsubsection*{Bill to Act}
\begin{itemize}
    \item Bil becomes an act of parliament
    \item Commencement
    \item When does an Act officially come into force?
        \begin{itemize}
            \item a date or event specified in Act
                \begin{description}
                    \item[Note:] like a declaration of a war.  
                \end{description}
            \item The date of royal assent (default for QLD)
            \item 28 days from date of royal assent (default for Cth)
        \end{itemize}
\end{itemize}

\subsubsection*{Legislation}
Act -> Regulations(details)
2009 and 20019, the number of new legislation is doubled. 

\begin{itemize}
    \item Acts as Passed: past versions.
\end{itemize}

\subsection*{Source of Law}
Case Law (C/L) is the judge's explanation of legislation. 

\subsection*{For a case}
\begin{itemize}
    \item Jurisdiction 
        \begin{description}
            \item[Note:] Cth legislation is higher than state legislation. Sometimes, the coverage of the legislation is different. 
        \end{description}

    \item Time

    \item capacity 
        \begin{description}
            \item individual
            \item we can also check court
        \end{description}
\end{itemize}

\subsection*{Tips}
\begin{itemize}
    \item Have a look at the practice quiz
    \item 
\end{itemize}


\section*{Workshop Three}
\date{March 19th 2024}
\subsection{Exam WK4}
\begin{itemize}
    \item MCQ 15 questions
    \item 50 min
\end{itemize}



\subsection*{Common Law -- System}
\begin{itemize}
    \item it is based on the doctrine of the precedent. 
    \item It is an adversarial system ( represented to argue the case. )
    \item The judge only determines and makes decisions.
\end{itemize}


\subsection{Civil Law -- System}
\begin{itemize}
    \item it is an inquisitorial process (judge running the show)
    \item it is codified. Little or no doctrine of precedents. 
\end{itemize}

\subsection{Autocratic Law}
\begin{itemize}
    \item no separation of the power. 
\end{itemize}

\subsection*{Common Law -- Source of the Law}

\begin{itemize}
    \item Cession -- Treaty making (Canada/NZ)
    \item Settlement
    \item Conquest -- St. Petersburg.
\end{itemize}


\subsection*{Australian Legal System}
\begin{itemize}
    \item 1968 Australia act -- cut the tie with the UK. HCA will be the terminal court. 
    \item Westminster system -- common law
        \begin{itemize}
            \item constitution:  two levels of the government. (Fed and States, territory.)
                \begin{description}
                    \item[Note:] Section 51 describes the dividend of the power. 
                \end{description}
            \item Crown is the head of the states
            \item separation of the power. 
        \end{itemize}
    
   
\end{itemize}


\subsection{term}
\begin{itemize}
    \item C/L ( common law, case law) means the source of the law.
    \item system of law 
        \begin{itemize}
            \item doctrine
            \item precedent
        \end{itemize}
    \item In a criminal case, v reads as against
        \begin{description}
            \item evidence has to be beyond reasonable doubt
        \end{description}
    \item In a civil case, v reads as and.
        \begin{description}
            \item evidence needs to be a balance of probability. (More likely than not)
        \end{description}
    \item The First Party is either
        \begin{itemize}
            \item The plaintiff - party who initiated the proceedings
            \item The applicant - making an application to the court
            \item An Appellant - appealing a decision of a lower court
        \end{itemize}
    \item The Second Party is either
        \begin{itemize}
            \item The plaintiff - party who initiated the proceedings
            \item The applicant - making an application to the court
            \item An Appellant - appealing a decision of a lower court
        \end{itemize}
    \item R
        \begin{itemize}
            \item R -Regina
            \item square brackets vs. round brackets 
            \begin{description}
                \item[Square brackets:]
                \item[Round brackets: ]
            \end{description}
        \end{itemize}
\end{itemize}



\section*{Workshop Four}
\date{March 26th 2024}

\subsection*{Natural}
how should the law look like
\subsubsection*{Term}
\begin{itemize}
    \item Natural Law
        \begin{description}
            \item[Definition:] Law is derived from higher, aspirational principles and values. \footnote{Michelle Sanson and Thalia Anthony,\textit{Connecting with Law}(Oxford,5ed,2022)323}
        \end{description}

\end{itemize}

\subsubsection*{Key Person}
\begin{itemize}
    \item Aristotle (384-322 BCE)
        \begin{description}
            \item[Definition:] Law is derived from higher, aspirational principles and values. \footnote{Michelle Sanson and Thalia Anthony,\textit{Connecting with Law}(Oxford,5ed,2022)323}
        \end{description}
    \item Key Feature
        \begin{description}
            \item[1]Law is not made but discovered by observation and contemplation of the nature of things
            \item[2] Must conform to the innate natural law. i.e., Aristotle believed that defiance of unjust laws in the pursuit of natural law is morally justified.
            \item[3] equity attempts to correct the deficiencies of the common law as it is applied in a particular instance
            \item[4] morality has a high priority. 
            \item[5] Freedom from arbitrary treatment (northern territory thing)
        \end{description}
\end{itemize}

\subsection*{Positivism}
Positivism is about what is a law. It is about the rules. It influences the reasoning approach of formalism. 
\begin{itemize}
    \item Thomas Hobbes 
        \begin{description}
            \item[Social Contract:] If the contract is breached, the power goes back to citizens, and the state is less controlled. 
        \end{description}
    \item john lock
    \item HLA Hart
        \begin{description}
            \item[Definition:] Law is derived from higher, aspirational principles and values. \footnote{Michelle Sanson and Thalia Anthony,\textit{Connecting with Law}(Oxford,5ed,2022)323}
        \end{description}
    \item Key Feature
        \begin{description}
            \item[1] social contract. 
            \item[2] Law is what it is, not what it oughts to be. 
            \item[3] Public and private separation
        \end{description}
\end{itemize}



\subsection*{Liberalism}
concerned with liberty 
\begin{itemize}
    \item Thomas Hobbes 
        \begin{description}
            \item[Social Contract:] If the contract is breached, the power goes back to citizens, and the state is less controlled. 
        \end{description}
    \item john lock
    \item HLA Hart
        \begin{description}
            \item[Definition:] Law is derived from higher, aspirational principles and values. \footnote{Michelle Sanson and Thalia Anthony,\textit{Connecting with Law}(Oxford,5ed,2022)323}
        \end{description}
    \item Key Feature
        \begin{description}
            \item[1] individualism, the community is regarded simply as some of its parts. 
            \item[2] The government's power is limited. The government is only going to intervene to avoid harm. 
            \item[3] equality and right. 
            \item[4]
        \end{description}

    \subsubsection*{Idea}
        \begin{itemize}
            \item liberty
                \begin{description}
                    \item[Note:] We are talking about limiting governments over the people, we talk about the market, and the individual is free from government international strain. The role of society and the state/government is limited, and they only intervene to avoid harm. 
                    \item[Focus:] Individuals have rights and are entitled to have them protected by law
                \end{description}
                \begin{itemize}
                    \item positive liberty
                        \begin{itemize}
                            \item 
                            
                        \end{itemize}
                    \item classic liberty
                        \begin{itemize}
                            \item statutory interpretation 
                                \begin{description}
                                    \item[Note:] fundamental principle: human right 
                                \end{description}
                                \item contract law (K-law)
                        \end{itemize}
                \end{itemize}
            \item public v private
                \begin{description}
                    \item[Note:] How much government can intervene in private life. 
                \end{description}

            \item individualism
                \begin{description}
                    \item[Note:] not only the human individual, but a corporation is a legal individual. Everyone is equal. They talk about free markets. 
                \end{description}


            \item equality
                \begin{description}
                    \item[Note:]
                \end{description}

                \begin{itemize}
                    \item Formal: everyone is equal. 
                    \item Substantive: state interventions are required to achieve real equality. 
                        \begin{description}
                            \item[i.e.] social security 
                        \end{description}
                \end{itemize}

            \item Rights
                 \begin{description}
                    \item[Note:] rights conflicts apply harm principle. 
                \end{description}

                \begin{itemize}
                    \item Deontological : normative jurisprudence. moral basis for exercising rights. Duty. 
                    \item teological: the form of analytical jurisprudence. Here, we can apply the utilitarian principle (Jeremy Benthem). This approach focuses on consequences, not morality. 
                \end{itemize}

        \end{itemize}
    
\end{itemize}



\subsection*{Introduce Yourself}
Good Morning, Your Honor,

If it pleases the court, my name is   Xiaoyu Wu, Wu, first initial X. 

I appear for :[The respondent/applicant/defendant/Plaintiff]

\subsubsection*{Definition}
\begin{itemize}
    \item The rule of law: No one above the law
    \item responsible government: responsible to the people, people elect ministers; ministers heads of executive
    \item separation of the power: division of power to avoid corruption
    \item parliamentary sovereignty: source of the law (c/l, statute), develop the law appropriately. 
    \item HLA Hart: the law is a law if the right process makes the law. 
    
\end{itemize}


\section*{Workshop Five}
\date{April 9th 2024}

\subsection*{Re Liveri [2006] QCA 152}

\begin{itemize}
    \item What is the ratio in this case?
        \begin{description}
            \item[paragraph:]21
            \item[Detail:]The findings against the respondent involve serious plagiarism committed more than once. At relevant times, she was a person of mature years – 25 and 27 years old. Her unwillingness, subsequently, to acknowledge that misconduct establishes a lack of genuine insight into its gravity and significance: for present purposes, where the Court is concerned with fitness to practice, that aspect is at least as significant as the academic dishonesty itself. It could not presently be concluded the applicant is fit for admission as a legal practitioner.
        \end{description}
        
    \item What principle of law does this case establish?
        \begin{description}
            \item[A:]
        \end{description}
\end{itemize}

\subsection*{Borhani v Legal Practitioners Admissions Board[2013] QCA 14}
\begin{itemize}

\item What is the ratio in this case?
        \begin{description}
            \item[paragraph:]17-18
            \item[Detail:]The findings against the respondent involve serious plagiarism committed more than once. At relevant times, she was a person of mature years – 25 and 27 years old. Her unwillingness, subsequently, to acknowledge that misconduct establishes a lack of genuine insight into its gravity and significance: for present purposes, where the Court is concerned with fitness to practice, that aspect is at least as significant as the academic dishonesty itself. It could not presently be concluded the applicant is fit for admission as a legal practitioner.
        \end{description}
        
    \item What principle of law does this case establish?
        \begin{description}
            \item[A:]
        \end{description}
\end{itemize}


\subsection*{Reception of English Laws}
\begin{itemize}
    \item cession -- treaty (NZ)
    \item settlement -- Mabo(No.2), Blackstone
    \item conquest -- 
\end{itemize}

\subsubsection{Case Law/Common Law}
\begin{itemize}
    \item Rule of Law (Equality)
    \item Separation of Power
        \begin{description}
            \item[Note:] decentralize the power
        \end{description}
        \begin{itemize}
            \item Executive
            \item Parliament
            \item Judaical
                \begin{description}
                    \item[Note:] judge can be Formalist, activists. 
                \end{description}
        \end{itemize}
    \item Doctrine of precedents
\end{itemize}

\subsubsection*{Westminster system}

\paragraph*{Representation Government}


\paragraph*{Constitution}
Both the Commonwealth \& State have a constitution (pluralism). Important section in the constitution

\begin{itemize}
    \item s72  Judges' appointment, tenure, and remuneration
        \begin{description}
            \item[] independent judiciary
            \item[Description:]
        \end{description}
    \item s80 
        \begin{description}
            \item[] trial by jury
        \end{description}
    \item s71 Judicial power and Courts (pluralism)
        \begin{description}
            \item[Description:]The judicial power of the Commonwealth shall be vested in a Federal Supreme Court, to be called the High Court of Australia, and in such other federal courts as the Parliament creates, and in such other courts as it invests with federal jurisdiction. The High Court shall consist of a Chief Justice and many other Justices, not less than two, as the Parliament prescribes.
            \item[Short:]HCA jurisdiction
        \end{description}
    \item 
\end{itemize}

\paragraph{admission process}
In QLD, the process is governed by the Legal Profession Act 2007 (QLD):
\begin{itemize}
    \item Section 9
    \item Section 30 + 9 suitability
    \item Section31
    \item Section 34
    \item section 35
\end{itemize}


\section*{Week Six Workshop}

\subsection*{Judgecraft}
\begin{minipage}{\textwidth}
\begin{tabular}{lllll}
     Name of judge & Reasoning approach & Interpretation & Legal Philosophy & Comment  \\
     \hline \\
     name   & Formalism\footnote{1.literal approach, 2. method \&reason, 3. The judge should not change the law, 4. protect the judge by focusing on rules, 5. sticking with the rules, 6. A way of limiting arbitrary opinion} & literal\footnote{purely words} & Positivism & {} \\
     {} & Realism \footnote{1.the impact on society, 2. Mention changes but not act on changing laws, 3. Justice in Mabo [No2].4. anti formalism. 5. acknowledge the judges' basis 6. doctrine of the precedent 7. reflecting the life and situation}& Purposive\footnote{text and context in written law and we need to understand the purpose of law.} & Natural Law &{}\\
     {} & Activism \footnote{1. Justice Kerbie in Mabo [No2] 2. judge role is promoting justice 3. less bind with formal constraints 4. fills the gap in the law where parliament does not act.} &{} &liberalism & {}
\end{tabular}
\end{minipage}

\subsection*{Answer Structure}

\begin{itemize}
    \item What is the purpose of this? -- reference the law
    \item Who is your audience? --
    \item Use Headings to Guide your Answer
        \begin{itemize}
            \item Parties/jurisdiction/commencement
            \item Issue
                \begin{description}
                    \item[Example:] is xxx a fit and proper person to be admitted as a legal practitioner in Queensland 
                \end{description}
            \item Law 
                \begin{description}
                    \item[Example:] Re v AJG and section 31
                \end{description}
            \item Apply the rule
                \begin{description}
                    \item[Note:] Use the case precedents to provide authority \& context 
                \end{description}
        \end{itemize}
\end{itemize}

\section*{Week Seven Workshop}

\subsection*{Hypothetical}

\subsubsection*{Parties}
The applicant's application to be admitted as a solicitor in QLD has been denied. 
\begin{itemize}
    \item Applicant
    \item Respondent: Legal Practitioners Admissions Board
    \item Jurisdiction: QLD
\end{itemize}

\subsubsection*{Issue}
The applicant's application to be admitted as a solicitor in QLD has been denied due to academic dishonesty. The main legal issue is whether the applicant is a proper and suitable person for a legal profession under the Legal Profession Act 2007 (QLD) s30.

\subsubsection*{Law}
The Legal Profession Act 2007 (QLD) s 30
\begin{center}
    \begin{enumerate}
        \item A person is suitable for admission under this Act as a legal practitioner only if the person is a fit and proper person.
        \item  In deciding if the person is a fit and proper person, the Supreme Court must consider—
            \begin{enumerate}
                \item each of the suitability matters in relation to the person to the extent a suitability matter is appropriate; and
                \item other matters that the court considers relevant.
            \end{enumerate}
        \item  However, the Supreme Court may consider a person suitable for admission under this Act as a legal practitioner despite a suitability matter because of the circumstances relating to the matter.
    \end{enumerate}
\end{center}

Section 9 has detailed the suitability matters. 

The following cases set up the precedent for explaining suitability issues. 
\begin{itemize}
    \item Liveri, Re [2006] QCA 152; BC200603197
    \item Borhani v Legal Practitioners Admissions Board [2013] QCA 014; BC201300532
    \item AJG, Re [2004] QCA 088; BC200401539
    \item Re Onyeledo (No 2) [2016] NTSC 68; BC201610656
\end{itemize}


\subsubsection*{}

\subsection*{Ethic \& Morality}
\begin{itemize}
    \item Deontological ethics (from Greek: \textgreek{d\'e\ on}, `obligation, duty' + \textgreek{l\'ogos}, `study')
        \begin{description}
            \item[Note:] it is the normative ethical theory \footnote{Normative ethics is the study of ethical behavior and is the branch of philosophical ethics that investigates questions regarding how one ought to act, in a moral sense.} that the morality of an action should be based on whether that action itself is right or wrong under a series of rules and principles, rather than based on the consequences of the action. It focuses on duties.
        \end{description}
    \item Teleological(from \textgreek{t\'elos}, \emph{telos}, `end', `aim', or `goal', and \textgreek{l\'ogos}, \emph{logos}, `explanation' or `reason')
        \begin{description}
            \item[Note:] it is a branch of causality giving the reason or an explanation for something as a function of its end, its purpose, or its goal, as opposed to as a function of its cause. It focuses on the consequences of actions. 
        \end{description}
    \item Virtue Ethics (from Greek: \textgreek{>aret'h})
        \begin{description}
            \item[Note:] It was advocated by Aristotle, with some aspects being supported by Saint Thomas Aquinas, focusing on the inherent character of a person rather than on specific actions. There has been a significant revival of virtue ethics since the 1950s through the work of such philosophers as G. E. M. Anscombe, Philippa Foot, Alasdair MacIntyre, and Rosalind Hursthouse.
        \end{description}
\end{itemize}

\begin{minipage}{\textwidth}
\begin{tabular}{p{1cm}p{3cm}p{3cm}p{4cm}}
     Rule No. & Detail/Proposition & Example/Application to the rule & Comments/ Case or Legs  \\
     \hline
     3 & Paramount duty to the court and the administration of justice & Primary Duty to the Court and the system of justice & Not to mislead the duty of candour \\

     4 & Other Fundamental Duties & {} & Wrt client Gifts \\
     5 & Dishonest and disreputable conduct & ``do not bring profession into disrepute'' & {}\\
\end{tabular}
\end{minipage}

\section*{Workshop Week Eight}
\date{April 30th 2024}

\subsection*{Statutory Interpretation}

\subsubsection*{common law historical approach}
The historical approach to statutory law in common law approach has the following rules: 
\begin{itemize}
    \item Mischief Rule (current practice)
        \begin{description}
            \item[Case] Smith v Hughes [1960] 2 All ER 859
        \end{description}
    \item Golden Rule
        \begin{description}
            \itm[Note:] Overcome issues with the literal rule where the result led to “absurdity or inconsistency”
            \item[Case:]Grey v Pearson (1857) 6 HL Cas 61
        \end{description}
    \item literal rule
        \begin{description}
            \item[Authority:] Amalgamated Society of Engineers v Adelaide Steamship Company Pty ltd (
        \end{description}
    \item Justification
        \begin{description}
            \item[Authority:] Nelson v Nelson (1995) 185 CLR 538, 555 
        \end{description}
\end{itemize}

\subsubsection*{Purposive Approach}
the modern approach
\begin{itemize}
    \item refers to the statutory approach contained in amendments to acts interpretations act
    \item reading the statute as a whole
    \item significantly more flexible, consistent and coherent
    \item HCA deliberately identifying adopting a modern approach
        \begin{description}
            \item[Authority:]CIC Insurance Ltd Backstown Football Club Ltd (1997) 187 CLR 384 
            \item[Authority:] Project Blue Sky v Australian Broadcasting Authority
        \end{description}
    \item AIA(Cth) s 15AA
        \begin{description}
            \item[Note:] An interpretation that would achieve the purpose or object of the act is to be preferred to each other interpretation. Inserted in 1981
        \end{description}
    \item AIA(QLD) s14A
        \begin{description}
            \item[Note:] The interpretation that would best achieve the purpose of the Act is to be preferred to any other interpretation 
            \item[Attention:] 
        \end{description}
    \item Intrinsic 
        \begin{description}
            \item[Autority:] Fed Cas Employees 
            \item[e.g] AIA,
            \item[Note:]footnote can only be used at cth level. 
        \end{description}
    \item Extrinsic 
        \begin{description}
            \item[Note:]Only when you find intrinsic materials are not able to meet the needs for a purposive approach, you go to extrinsic 
            \item[e.g.] Hansard, explain notes (QLD), explain memo (Cth)
            \item[Authority:]CIC b Bankstwon; AIA
        \end{description}
\end{itemize}

State legislation is effective on the date of assent. Cth legislation will have a 28-day buffer from the date of assent. 

\section*{Week Nine Workshop}



\section*{Week Ten Workshop -- Statutory Interpretation Part 3}

\subsection*{Statutory Interpretation Process}
\begin{enumerate}
    \item Parties
    \item Capacity
    \item Jurisdiction (State/Cth and the level of court in the hierarchy)
    \item Time/Commencement 
        \begin{description}
            \item[Note:] Need to provide information about the legislation assent date and the event date.  
        \end{description}
    \item Issue
        \begin{itemize}
            \item Person A's main issue
                \begin{itemize}
                    \item Subissues
                \end{itemize}
            \item Person B's main issue
                \begin{itemize}
                    \item Subissues
                \end{itemize}
        \end{itemize}
    \item Statutory Interpretation
        \begin{itemize}
            \item C/L SI
            \item Purposive\footnote{AIA (Cth s15AA, QLD s14)}
                \begin{itemize}
                    \item Intrinsic
                    \item Extrinsic
                \end{itemize}
        \end{itemize}
\end{enumerate}

\subsection*{Statutory Interpretation Tool}
\begin{itemize}
    \item Common Law Presumptions
        \begin{itemize}
            \item Presumption Regarding fundamental rights
                \begin{description}
                    \item[Note:] Statutes cannot be interpreted in such a way as to interfere with fundamental rights unless there is a clear intention to the contrary
                    \item[Autority:] ReBolton, Ex parte Beane (1987) 162 CLR 514, 523) 
                    \item[e.g:] C/L presumptions against retrospectively, however, this act poses a rebuttal, and the act applies to person A. 
                \end{description}
        \end{itemize}
\end{itemize}

\subsection*{Question 4}
\begin{itemize}
    \item key points
        \begin{itemize}
            \item Latin Maxims: ejusdem generis
            \item Rebuttal of c/l presumptions.
        \end{itemize}
\end{itemize}
